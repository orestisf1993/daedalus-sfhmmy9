\subsection{Ισχύς Ζεύξης Συσκευών}
Για να μπορούμε να μετρήσουμε την απόσταση των δύο συσκευών μας, είναι 
απαραίτητο να γνωρίζουμε την ισχύ της ζεύξης των δύο συσκευών. Όπως έχουμε 
ήδη αναφέρει το ένα bluetooth χρησιμοποιείται για την επικοινωνία των συσκευών και 
το άλλο για την μέτρηση της ισχύος μεταξύ των δύο συσκευών, το οποίο ονομάζεται 
rssi και το android api το επιστρέφει σε μονάδες dB. 

Μέσα από διάφορα πειράματα που εκτελέστηκαν κατά την διάρκεια του διαγωνισμού 
φτάσαμε στα παρακάτω συμπεράσματα:
\begin{itemize}
\item Η απόσταση στην οποία πρέπει να ενεργοποιείται ο κίνδυνος είναι 5 με 6  
μέτρα
\item Σε απόσταση 4 με 5 μέτρα πρέπει να βρισκόμαστε στην κατάσταση στην οποία 
υπάρχει προειδοποίηση για πιθανό κίνδυνο
\end{itemize}   

Οπότε καταλήξαμε πως για τιμές του rssi μεγαλύτερες από -75dB βρισκόμαστε σε 
ασφαλή κατάσταση.
Για τιμές μεταξύ -75dB και -80dB βρισκόμαστε σε κατάσταση πιθανού 
κινδύνου, ενώ σε τιμές μικρότερες από -80dB σε κατάσταση κινδύνου.
